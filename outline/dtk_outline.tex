%        File: dtk_outline.tex
%
\documentclass[letterpaper]{article}
\usepackage[top=1.0in,bottom=1.0in,left=1.0in,right=1.0in]{geometry}
\usepackage{appendix}
\usepackage{verbatim}
\usepackage{amssymb}
\usepackage{graphicx}
\usepackage{longtable}
\usepackage{amsfonts}
\usepackage{amsmath}
\usepackage{algpseudocode} 
\usepackage[usenames]{color}
\usepackage[
  naturalnames = true, 
  colorlinks = true, 
  linkcolor = black,
  anchorcolor = black,
  citecolor = black,
  menucolor = black,
  urlcolor = blue
]{hyperref}
\usepackage{listings}
\usepackage{textcomp}
\definecolor{listinggray}{gray}{0.9}
\definecolor{lbcolor}{rgb}{0.9,0.9,0.9}
\lstset{
  backgroundcolor=\color{lbcolor},
  tabsize=4,
  rulecolor=,
  language=c++,
  basicstyle=\scriptsize,
  upquote=true,
  aboveskip={1.5\baselineskip},
  columns=fixed,
  showstringspaces=false,
  extendedchars=true,
  breaklines=true,
  prebreak =
  \raisebox{0ex}[0ex][0ex]{\ensuremath{\hookleftarrow}},
  frame=single,
  showtabs=false,
  showspaces=false,
  showstringspaces=false,
  identifierstyle=\ttfamily,
  keywordstyle=\color[rgb]{0,0,1},
  commentstyle=\color[rgb]{0.133,0.545,0.133},
  stringstyle=\color[rgb]{0.627,0.126,0.941},
}

%%---------------------------------------------------------------------------%%
\author{Stuart R. Slattery
  \\ \href{mailto:sslattery@wisc.edu}{\texttt{sslattery@wisc.edu}}
}

\date{April 18, 2012}
\title{Design and Implementation of the Data Transfer Kit}
\begin{document}
\maketitle
\newpage

%%---------------------------------------------------------------------------%%
\section{Introduction}
For most multiphysics simulations, some form of data transfer between
physics codes is required. In general, data transfer can be performed
through an ad-hoc coupling of the codes with custom software designed
for the specific simulation. However, it is useful to consider a
component with an associated interface for data transfer
operations. This common interface, to be fulfilled by each code that
desires to participate in data transfer operations using this
component, should reduce repeated effort and define generic and
reusable software that can be applied to many forms of multiphysics
simulation.

The data transfer package aims to define such an interface for
coupling in addition to providing concrete implementations of
algorithms and communication patterns for parallel data transfer using
the information acquired through this interface. It should not be the
aim of the data transfer package to eliminate other forms of
multiphysics data transfer. Rather, it should attempt to define a
useful interface that provides value by simplifying the data transfer
process and reducing the barrier of entry for a physics code to
participate in coupling.

This document discusses the design requirements of this package, named
the Data Transfer Kit (DTK), the resulting interface specification,
the implementation of a basic transfer algorithm using this interface,
and instructions on building DTK.

%%---------------------------------------------------------------------------%%
\section{Design Requirements for the Data Transfer Kit}
The Data Transfer Kit should fulfill the following design
requirements:

\begin{itemize}
\item Be able to correctly transfer data between multiple codes that
  have implemented the interface.
\item Be independent of the underlying data structures of the physics
  codes using the interface.
\item Be aware of the potential mesh entity structures that are
  associated with the fields being transferred (i.e. vertices, faces,
  elements, etc.).
\item Be aware of the parallel communication objects that operate the
  codes being coupled.
\item Be able to generate a parallel topology map using information
  made available through the interface.
\item Be able to transfer multiple fields using the topology map and
  information made available through the interface.
\end{itemize}

%%---------------------------------------------------------------------------%%
\section{Preliminary Design of the Data Transfer Kit}
As part of the requirements listed in the previous section, it is
necessary to define an interface for the Data Transfer Kit that will
allow access to information held by each physics code being coupled in
order to meet the coupling algorithm requirements in concrete data
transfer implementations.

In its current form, the interface to the Data Transfer Kit exists as
two abstract interfaces contained by the {\sl DataSource} and the
{\sl DataTarget} classes. At least one of these interfaces should be
fulfilled by each physics code that desires to participate in coupling
under this framework by creating its own implementation of the {\sl
  DataSource} interface for supplying data to the Data Transfer Kit
and the {\sl DataTarget} interface for receiving data from the
Data Transfer Kit. By implementing these, a code will provide the
necessary information to implement coupling algorithms. 

The following sections provide the {\sl DataSource} and {\sl
  DataTarget} abstract interfaces. All methods defined by these
interfaces are pure virtual and must therefore be defined by all
implementations. Three template parameters are defined for the
interfaces, the scalar data type to be transferred, the handle type
associated with mesh entities (commonly int or long int), and the
coordinate type for defining spatial locations (commonly double or float).

The Teuchos package of Trilinos is used for memory management,
imposing the requirement on interface implementations to use these
utilities. The underlying physics package need not implement these
memory management tools, however, they must provide an interface
implementation with these utilities. See \cite{Bartlett_2010} for more
details on the Teuchos memory management scheme.

Communication is handled by the Teuchos abstract communication
interface, allowing for various architectures to be supported. In the
instance of MPI-based communication, any communication object
implementing MPI primitives (e.g. MPI\_SEND, MPI\_RECEIVE, etc.) can
be used. 

As an initial attempt to create mesh aware data structures, a simple
point object has been implemented as a container for coordinates and
an associated identifier handle within the mesh subpackage. These
interfaces also impose the requirement that this point data structure
be used in the implemenations.

The following sections outline both interfaces and details on their
requirements. 

%%---------------------------------------------------------------------------%%
\subsection{Data Source Interface}
The {\sl DataSource} interface consists of the following pure
virtual methods to be implemented by the client application.

\paragraph{\sl DataSource::get\_source\_comm()}
This method requires the client application to return a reference
counting pointer (Teuchos::RCP objects) to a const Teuchos
communicator object. For the case of an MPI implementation, this is
not required to be MPI\_COMM\_WORLD.

\paragraph{\sl DataSource::is\_field\_supported()}
This method checks whether or not a particular field, indicated by
name with a string, is implemented for transfer by the client. Given
this string, the client should return true if the field is supported,
false if not.

\paragraph{\sl DataSource::is\_local\_point()}
Given a point object, the client returns true if this point resides in
the local domain and should therefore be mapped, and false if not. As
this is the only method in the interface that supplies target points
to the source application, it is recommended that if the point handles
and/or coordinates are required by the client for data transfer, that
they be cached during the implementation of this method. 

\paragraph{\sl DataSource::are\_local\_points()}
An addtional point query method, behaves identically to {\sl
  DataSource::is\_local\_point()}, with the query operation performed
for a series of points. A default implementation of this method is
provided for those who only implement the {\sl is\_local\_point}
method.

\paragraph{\sl DataSource::get\_source\_data()}
For a given field identified by a string, return a const persisting
view (a Teuchos::ArrayRCP) of the data corresponding to the target
points found in the local domain through the {\sl get\_points} method.

\paragraph{\sl DataSource::get\_global\_data\_data()}
For a given field identified by a string, return a single global data
value.

The following listing provides the {\sl DataSource} abstract
interface. 

\begin{lstlisting}
template<class DataType, class HandleType, class CoordinateType, int DIM>
class DataSource : Teuchos::Describable
{
  typedef int                                        OrdinalType;
  typedef Point<DIM,HandleType,CoordinateType>       PointType;
  typedef Teuchos::Comm<OrdinalType>                 Communicator_t;
  typedef Teuchos::RCP<const Communicator_t>         RCP_Communicator;

  virtual RCP_Communicator get_source_comm() = 0;

  virtual bool is_field_supported( const std::string &field_name ) = 0;

  virtual bool is_local_point( const PointType &point ) = 0;

  virtual const Teuchos::ArrayRCP<bool> 
  are_local_points( const Teuchos::ArrayRCP<PointType> points );

  virtual const Teuchos::ArrayRCP<DataType> 
  get_source_data( const std::string &field_name ) = 0;

  virtual DataType 
  get_global_source_data( const std::string &field_name ) = 0;
};
\end{lstlisting}

%%---------------------------------------------------------------------------%%
\subsection{Data Target Interface}
The {\sl DataTarget} interface consists of the following pure
virtual methods to be implemented by the client application.

\paragraph{\sl DataTarget::get\_target\_comm()}
This method requires the client application to return a Teuchos::RCP
to a const reference of its communicator object. For the case of an
MPI implementation, this is not required to be MPI\_COMM\_WORLD.

\paragraph{\sl DataTarget::is\_field\_supported()}
This method checks whether or not a particular field, indicated by
name with a string, is implemented for transfer by the client. Given
this string, the client should return true if the field is supported,
false if not.

\paragraph{\sl DataTarget::get\_target\_points()}
For a given field identified by a string, return a Teuchos::ArrayRCP
(a persisting view) of all the points to which data should be mapped
in the local target domain. All point handles are required to be
globally unique.

\paragraph{\sl DataTarget::get\_target\_data\_space()}
For a given field identified by a string, return a non-const
persisting view (Teuchos::ArrayRCP) of the data vector associated with
the points provided by the {\sl set\_points} method. This view will be
used to write data into the target.

\paragraph{\sl DataTarget::set\_global\_target\_data()}
For a given field identified by a string, a single global data value
is provided.

The following listing provides the {\sl DataTarget} abstract
interface. 

\begin{lstlisting}
  template<class DataType, class HandleType, class CoordinateType, int DIM>
  class DataTarget : public Teuchos::Describable
  {
    typedef int                                        OrdinalType;
    typedef Point<DIM,HandleType,CoordinateType>       PointType;
    typedef Teuchos::Comm<OrdinalType>                 Communicator_t;
    typedef Teuchos::RCP<const Communicator_t>         RCP_Communicator;

    virtual RCP_Communicator get_target_comm() = 0;

    virtual bool is_field_supported( const std::string &field_name ) = 0;

    virtual const Teuchos::ArrayRCP<PointType> 
    get_target_points( const std::string &field_name ) = 0;

    virtual Teuchos::ArrayRCP<DataType> 
    get_target_data_space( const std::string &field_name ) = 0;

    virtual void set_global_target_data( const std::string &field_name,
    const DataType &data ) = 0;
  };
\end{lstlisting}

%%---------------------------------------------------------------------------%%
\subsection{Copy Operator}
The relationship between a particular {\sl DataSource} and {\sl
  DataTarget} implementation is contained by the {\sl CopyOperator}
object. Both mapping and copy operations between the {\sl DataSource}
and {\sl DataTarget} are handled by the {\sl CopyOperator}. The
following methods are available for the {\sl CopyOperator} object.

\paragraph{\sl CopyOperator::CopyOperator()}
The constructor for this object requires the communicator on which to
operate. At a minimum, this should correspond to the union of the
target and source communicators. Again, for MPI implementations this
is not limited to MPI\_COMM\_WORLD. Rather, it is only required that
this communicator span the process ID space that encompasses both
communicators provided by the source and target interfaces. In
addition, a string defining the name of the field, reference counting
pointers to the two interface implementations, and an optional boolean
indicating whether or not this is a scalar field. The default value of
this boolean is false, indicating that the field is distributed.

\paragraph{\sl CopyOperator::create\_copy\_mapping()}
Generate the parallel mapping between the source and target
applications for the copy operation. This method has no arguments but
is required to be called before copy operations can be completed.

\paragraph{\sl CopyOperator::copy()}
Copy the data from the source application to the target
application. This method has no arguments.

The following listing gives an example at the driver level of how the
{\sl CopyOperator} object is used to transfer data between the two
interfaces. In this example, the scalar data type to be transferred is
double, handle type is int, and coordinate type is float.

\begin{lstlisting}
  Teuchos::RCP<DataSource<double,int,float> > source = 
  Teuchos::rcp( new DataSource_Implementation<double,int,float>() );

  Teuchos::RCP<DataTarget<double,int,float> > target = 
  Teuchos::rcp( new DataTarget_Implementation<double,int,float>() );

  CopyOperator<double,int,float> copy_op( MPI_COMM_WORLD, 
                                          "SOURCE_DISTRIBUTED_FIELD",
                                          "TARGET_DISTRIBUTED_FIELD",
                                          source, 
                                          target ); 
                                         
  copy_op.create_copy_mapping();
  copy_op.copy();
\end{lstlisting}

%%---------------------------------------------------------------------------%%
\section{Consistent Finite Element Interpolation}
The initial implementation of the Data Transfer Kit is focused on
performing consistent finite element interpolation of the following
form as outlined in \cite{Farhat_1998} \cite{Jiao_2004}. We define
$\phi_i$ to be the shape functions discretizing the source function
$f$ and $\psi_i$ to be the shape functions discretizing the target
function $g$. Given a set of target points, $t$, associated with $g$
and its mesh, and a set of source points $s$ associated with $f$ and
its mesh, we evaluate $f$ such that $g(t) = \sum_i \phi_i(t)
f(s)$. Here, we evaluate all components of the source basis $\phi$ at
the target point and apply to it the source function value at each
component location (typically the nodes of the mesh cells). It should
be noted however, that this scheme is not strictly conservative as
outlined in Appendix C of \cite{Jiao_2004}. We therefore provide the
set of {\sl global\_data} methods as a means to transfer global scalar
data which can then be applied in a rebalance to enforce global
conservation of quantity (but not local conservation). 

From a design standpoint, the interface methods reflect the
requirements to perform this computation.  The method {\sl
  get\_target\_points} populates $t$ while {\sl is\_local\_point}
method determines the cell location (both geometric and parallel) of
each component in $t$ in order to perform the source basis function
evaluation $\phi_i(t)$ in the summation. The {\sl get\_source\_data}
method provides $\sum_i \phi_i(t)f(s)$ for each target point in $t$,
while {\sl get\_target\_data\_space} provides the space $g(t)$ in
which to apply the function evaluation data for each target point.

\subsection{Mapping Algorithm Implementation}
The following algorithm is used to generate a point based mapping
between between the source and target applications.

\begin{itemize}
  \item Get the target points from the target application to populate the target point set.
  \item Extract the globally unique handles from the target points.
  \item Build a distributed map of the target data vector using the handles as indices.
  \item Do a global reduction across all nodes to compute the maximum number
    of local target points.
  \item On each target node, build a buffer of points to send to the source nodes. This buffer
    will be of the length computed by the global reduction and padded with null points.

  \item For ( all target nodes ):
    \begin{itemize}
      \item Broadcast the buffer of target points for the current target
        node to all other nodes.

      \item For ( all points in the broadcasted buffer ):
        \begin{itemize}
        \item Query the source application domain with the point.

        \item If ( the point is found in the local source application domain ):
          \begin{itemize}
          \item Add the point handle to a distributed list.
          \end{itemize}
        EndIf
      \end{itemize}
      EndFor
    \end{itemize}
    EndFor

\item Build a distributed map of the source data vector using the
distributed list of handles found in the source domain as indices.
\item Setup a communication plan to transfer data between the source map
distribution and the target map distribution.
\end{itemize}

Per the requirement of the target application to provide globally
unique handles for all of its points, the end result of this mapping
is a distributed object with knowledge of how the target data is
distributed via the handle indices, and how the source
data is distributed (most likely differently) using those same handle
indices. Once these two distributions are known, the data transfer
operation is simply a copy of data from the source to the target with
each piece of source data sent to the target destination containing
its handle index determined by the communication plan. 

%%---------------------------------------------------------------------------%%
\section{Data Transfer Kit Error Handling Policy}
The Data Transfer Kit will rely on C++ exceptions as the
primary means of propagating error information to the user. Per the
discussion in \cite{Sutter_2004}, all runtime exceptions thrown by DTK
will inherit from std::runtime\_error, allowing users to isolate
errors of this type with try/catch blocks. Internally, these runtime
errors are divided into three categories; precondition exceptions that
arise from a function's inability to meet preconditions, postcondition
exceptions resulting from a function's failure to achieve a
postcondition, and invariant exceptions resulting from a function's
inability to establish or maintain an invariant. Throughout DTK, a
user should expect these exceptions to be used to test these three
fundamental conditions and for runtime exceptions to be thrown if they
are violated. In addition, when an exception is thrown, a user should
expect DTK to provide a message detailing why the exception was thrown.

For developers implementing the data transfer interfaces defined
within DTK, it is expected that these exceptions are used to verify
the three runtime conditions within their implementation. Doing so
will then present a consistent exception scheme at the boundaries of
DTK  for users implementing the DTK API that are neither DTK
developers or DTK interface developers. In addition, throwing these
exceptions in the interface implementations will facilitate error
propagation from the interface implementations to the user through the
various API calls. 

Internally, DTK relies on the Trilinos packages Teuchos and Tpetra,
both of which throw C++ exceptions through their own schemes. Calls to
these libraries within DTK will not be isolated with try/catch
blocks. Instead, their exceptions will be propagated through the API
boundary.

%%---------------------------------------------------------------------------%%
\section{Obtaining the Data Transfer Kit}
The Data Transfer Kit is version controlled with a git software
repository residing on the casl-dev server. Changes to this repository
are communicated to a CASL hosted and moderated mailing list. Access
requests to this list can be made at {\sl
  http://casl-dev.ornl.gov/mailman/listinfo/coupler-infrastructure}. In
addition, this repository includes the Data Transfer Kit into the VERA
continuous integration and nightly build process. Per the discussion
in the TriBITS lifecycle model \cite{Bartlett_2012}, the Data Transfer
Kit is designated as secondary stable (SS).

To begin, the VERA repository is required as the base infrastructure
for building and executing the Data Transfer Kit as a VERA
component. Due to its direct dependencies on the Trilinos packages
Teuchos and Tpetra, the second step in obtaining the Data Transfer Kit
is to obtain Trilinos. Using the fissile four development machines and
the associated development environment, the following steps are
required to build VERA with the Data Transfer Kit. For purposes of
testing this documentation, the machine u233 was used.

\paragraph{git clone /casl-dev/git-root/VERA VERA}
Clone a copy of the VERA git repository.

\paragraph{cd VERA}
Move into the VERA directory.

\paragraph{git clone /casl-dev/git-root/Trilinos Trilinos}
Clone a copy of the Trilinos git repository into the VERA home
directory. 

\paragraph{git clone /casl-dev/git-root/DataTransferKit DataTransferKit} 
Clone a copy of the DataTransferKit repository into the VERA home
directory.

%%---------------------------------------------------------------------------%%
\section{Building the Data Transfer Kit}
Once the Trilinos and DataTransferKit repositories have been cloned, both are
ready to be configured with VERA. The Data Transfer Kit uses the
Tribits build system based on cmake and configures as a VERA TPL.

%%---------------------------------------------------------------------------%%
\subsection{Configure}
Consider an out-of-source build in a {\sl /vera\_build} directory. The
following shell script, also found in the location {\sl
  /DataTransferKit/doc/build\_notes/sample\_cmake\_configure.sh},
configures Trilinos and the DataTransferKit for a parallel debug build
using the default MPI implementation in the fissile four development
environment. To date, the Data Transfer Kit has been tested with both
OpenMPI and MPICH2 implementations of the MPI specification on shared
and distributed memory hardware.

\begin{verbatim}
#!/bin/bash

cmake \
-D CMAKE_INSTALL_PREFIX:PATH=/vera_build              \
-D CMAKE_BUILD_TYPE:STRING=DEBUG                      \
-D CMAKE_VERBOSE_MAKEFILE:BOOL=ON                     \
-D TPL_ENABLE_MPI:BOOL=ON                             \
-D VERA_EXTRA_REPOSITORIES="Trilinos;DataTransferKit" \
-D Teuchos_ENABLE_EXTENDED:BOOL=ON                    \
-D VERA_ENABLE_DataTransferKit:BOOL=ON                \
-D DataTransferKit_ENABLE_TESTS:BOOL=ON               \
-D DataTransferKit_ENABLE_EXAMPLES:BOOL=ON            \
/VERA
\end{verbatim}

Aside from a standard VERA configuration with Trilinos, we designate
that the DataTransferKit is an extra VERA repository, the Teuchos package in
Trilinos should be extended to enable the communication utilities, and
the Data Transfer Kit should be enabled by VERA with tests and examples
on.

%%---------------------------------------------------------------------------%%
\subsection{Build}

After configuration, {\sl make} will build the library while {\sl make
  install} will install it in the specified prefix location.

%%---------------------------------------------------------------------------%%
\subsection{Test}
Unit tests in both serial and parallel have been implemented using the
Teuchos unit testing harness and are incorporated into the Tribits
build. Using the configuration above, after building the tests can be
invoked with ctest. If successful, you should see an output like that
presented below. Note below that the WaveDamper example to be later
discussed was executed as well.

\begin{verbatim}
stuart@beaker:~/software/builds/VERA$ ctest
Test project /home/stuart/software/builds/VERA
    Start 1: DataTransferKitCore_CommIndexer_test_MPI_4
1/7 Test #1: DataTransferKitCore_CommIndexer_test_MPI_4 .........   Passed    1.72 sec
    Start 2: DataTransferKitCore_Point_test_MPI_4
2/7 Test #2: DataTransferKitCore_Point_test_MPI_4 ...............   Passed    1.13 sec
    Start 3: DataTransferKitCore_Interfaces_test_MPI_4
3/7 Test #3: DataTransferKitCore_Interfaces_test_MPI_4 ..........   Passed    1.12 sec
    Start 4: DataTransferKitCore_CopyOperator_test_MPI_4
4/7 Test #4: DataTransferKitCore_CopyOperator_test_MPI_4 ........   Passed    1.28 sec
    Start 5: DataTransferKitCore_Advanced_Transfer_test_MPI_4
5/7 Test #5: DataTransferKitCore_Advanced_Transfer_test_MPI_4 ...   Passed    1.16 sec
    Start 6: DataTransferKitCore_Zero_Point_Proc_test_MPI_4
6/7 Test #6: DataTransferKitCore_Zero_Point_Proc_test_MPI_4 .....   Passed    1.15 sec
    Start 7: DataTransferKitCore_WaveDamperExample_MPI_4
7/7 Test #7: DataTransferKitCore_WaveDamperExample_MPI_4 ........   Passed    1.23 sec

100% tests passed, 0 tests failed out of 7

Label Time Summary:
DataTransferKit    =   8.79 sec

Total Test time (real) =  11.87 sec
\end{verbatim}

%%---------------------------------------------------------------------------%%
\section{Using the Data Transfer Kit}
To demonstrate using the Data Transfer Kit in a multiphysics
simulation, a simple example was constructed to show both
implementations of the {\sl DataSource} and {\sl DataTarget}
interfaces specified above and a multiphysics driver utilizing the
Data Transfer Kit to perform data transfer in an iterative manner. 

%%---------------------------------------------------------------------------%%
\subsection{WaveDamper Example}
Provided as an example in the DataTransferKit repository and enabled in the
above configuration, the WaveDamper example exists in the directory
{\sl /DataTransferKit/example/WaveDamper}. Two simple codes are coupled in the
simulation; a code called Wave that computes a one dimensional cosine
shape over the parallel domain and another code called Damper that
computes the amount by which the amplitude of an incoming function
should be decreased. They are coupled through a Picard iteration in
the following fashion until the amplitude of the wave is dampened
below a specified tolerance.

\begin{verbatim}
Wave computes its initial conditions.

while the amplitude of the wave computed by Wave is > tolerance:

    Wave transfers its solution to Damper.

    Damper computes the amount by which the amplitude of the Wave
    solution should be decreased.

    Damper transfers the computed dampening back to Wave.

    Wave applies the dampening to compute a new solution.
\end{verbatim}

Here, the convergence tolerance on the wave amplitude is specified as
1.0E-6 in the WaveDamper implementation.

%%---------------------------------------------------------------------------%%
\subsubsection{Wave Implementation}
The Wave code was implemented as shown in the following listings. Note
the methods included to provide access to various members relating to
data and the grid that have been added. These are similar to the types
of methods that will be required during the refactoring process
necessary for using the Data Transfer Kit. It should also be noted
that the Data Transfer Kit does not place any requirements on solve
methods. For the Wave code solve, note that we are additionally
returning the L2 norm of the vector we are operating on. In the
iteration sequenence presented in the driver code below, the magnitude
of the L2 norm will be used as a stopping criteria.

\begin{lstlisting}
  #include <vector>

  #include "Teuchos_RCP.hpp"
  #include "Teuchos_Comm.hpp"

  class Wave
  {
    private:
    
    Teuchos::RCP<const Teuchos::Comm<int> > comm;
    std::vector<double> grid;
    std::vector<double> f;
    std::vector<double> damping;

    public:

    Wave(Teuchos::RCP<const Teuchos::Comm<int> > _comm,
    double x_min,
    double x_max,
    int num_x);

    ~Wave();

    // Get the communicator.
    Teuchos::RCP<const Teuchos::Comm<int> > get_comm()
    {
      return comm;
    }

    // Get a const reference to the local grid.
    const std::vector<double>& get_grid()
    {
      return grid;
    }

    // Get a reference to the local data.
    std::vector<double>& get_f()
    {
      return f;
    }

    // Apply the damping to the local data structures from an external
    // source. 
    std::vector<double>& set_damping()
    {
      return damping;
    }

    // Solve the local problem and return the l2 norm of the local residual.
    double solve();
  };
\end{lstlisting}

\begin{lstlisting}
  #include "Wave.hpp"

  #include <algorithm>
  #include <cmath>

  Wave::Wave(Teuchos::RCP<const Teuchos::Comm<int> > _comm,
  double x_min,
  double x_max,
  int num_x)
  : comm(_comm)
  {
    // Create the grid.
    grid.resize(num_x);
    double x_size = (x_max - x_min) / (num_x);

    std::vector<double>::iterator grid_iterator;
    int i = 0;

    for (grid_iterator = grid.begin();
    grid_iterator != grid.end();
    ++grid_iterator, ++i)
    {
      *grid_iterator = i*x_size + x_min;
    }

    // Set initial conditions.
    damping.resize(num_x);
    std::fill(damping.begin(), damping.end(), 0.0);
    f.resize(num_x);
    std::vector<double>::iterator f_iterator;
    for (f_iterator = f.begin(), grid_iterator = grid.begin();
    f_iterator != f.end();
    ++f_iterator, ++grid_iterator)
    {
      *f_iterator = cos( *grid_iterator );
    }
  }

  Wave::~Wave()
  { /* ... */ }

  double Wave::solve()
  {
    // Apply the dampened component.
    double l2_norm_residual = 0.0;
    double f_old = 0.0;
    std::vector<double>::iterator f_iterator;
    std::vector<double>::const_iterator damping_iterator;
    for (f_iterator = f.begin(), damping_iterator = damping.begin();
    f_iterator != f.end();
    ++f_iterator, ++damping_iterator)
    {
      f_old = *f_iterator;
      *f_iterator -= *damping_iterator;
      l2_norm_residual += (*f_iterator - f_old)*(*f_iterator - f_old);
    }
    
    // Return the l2 norm of the local residual.
    return pow(l2_norm_residual, 0.5);
  }
\end{lstlisting}

%%---------------------------------------------------------------------------%%
\subsubsection{Damper Implementation}
The following listings give the implementation of the Damper
code. Again, note the methods used to provide access to the various
data structures needed by the Data Transfer Kit. These methods will be
used in the interface implementations given by the following sections.

\begin{lstlisting}
  #include <vector>

  #include "Teuchos_RCP.hpp"
  #include "Teuchos_Comm.hpp"

  class Damper
  {
    private:

    Teuchos::RCP<const Teuchos::Comm<int> > comm;
    std::vector<double> wave_data;
    std::vector<double> damping;
    std::vector<double> grid;

    public:

    Damper(Teuchos::RCP<const Teuchos::Comm<int> > _comm,
    double x_min,
    double x_max,
    int num_x);

    ~Damper();

    // Get the communicator.
    Teuchos::RCP<const Teuchos::Comm<int> > get_comm()
    {
      return comm;
    }

    // Get a reference to the local damping data.
    std::vector<double>& get_damping()
    {
      return damping;
    }

    // Get a reference to the local grid.
    std::vector<double>& get_grid()
    {
      return grid;
    }

    // Get the wave data to apply damping to from an external source.
    std::vector<double>& set_wave_data()
    {
      return wave_data;
    }

    // Apply damping to the local problem.
    void solve();
  };
\end{lstlisting}

\begin{lstlisting}
  #include "Damper.hpp"

  Damper::Damper(Teuchos::RCP<const Teuchos::Comm<int> > _comm,
  double x_min,
  double x_max,
  int num_x)
  : comm(_comm)
  {
    // Create the grid.
    grid.resize(num_x);
    double x_size = (x_max - x_min) / (num_x);

    std::vector<double>::iterator grid_iterator;
    int i = 0;

    for (grid_iterator = grid.begin();
    grid_iterator != grid.end();
    ++grid_iterator, ++i)
    {
      *grid_iterator = i*x_size + x_min;
    }

    // Set initial conditions.
    damping.resize(num_x);
    wave_data.resize(num_x);
  }

  Damper::~Damper()
  { /* ... */ }

  // Apply damping to the local problem.
  void Damper::solve()
  {
    std::vector<double>::iterator damping_iterator;
    std::vector<double>::const_iterator wave_data_iterator;
    for (damping_iterator = damping.begin(),
    wave_data_iterator = wave_data.begin();
    damping_iterator != damping.end();
    ++damping_iterator, ++wave_data_iterator)
    {
      *damping_iterator = *wave_data_iterator / 2;
    }
  }
\end{lstlisting}

%%---------------------------------------------------------------------------%%
\subsubsection{Wave Interface Implementations}
As given above by the algorithm used to couple Wave to Damper, two way
transfer is required and therefore both codes must implement the {\sl
  DataSource} and {\sl DataTarget} in order for this scheme to be
possible. The following listings in this section give the interface
implementations for the Wave code. For both interfaces, the scalar
transfer methods were implemented to perform no operations as this
example does not make use of them. Both implementations, though
templated to inherit from the pure virtual interface specifications,
do not use the template parameters. Rather, they explicitly specify in
the typdef block elements of type double to be transferred, int to be
used for point handles, and doubles to be used for coordinates.

To begin, the {\sl DataSource} implementation listing for the Wave
code has several elements that should be addressed. First, the
constructor in this implementation, and that of the {\sl DataTarget}
implementation, takes a pointer to the wave object that it will
operate on. For the {\sl get\_comm()} methods, a simple redirection is
used to grab the communicator from the local Wave object. 

For all transfers under this interface, the name of the field being
transferred is used to identify the appropriate operations to perform
in the interface. The {\sl Wave\_Data\_Source} implementation will
serve as the source for the WAVE\_FIELD and therefore the {\sl
  is\_field\_supported()} method designates this. In addition, all
other interface methods operating on a specific field will check for
this name before doing operations.

The {\sl is\_local\_point()} method checks the incoming point
against the local grid. If it is found to be a local grid point, true
is returned. Finally, the {\sl get\_source\_data()} method returns a
view of the Wave code solution, provided by a refactor method {\sl
  get\_f()}. 

\begin{lstlisting}
namespace DataTransferKit {

// DataSource interface implementation for the Wave code.
template<class DataType, class HandleType, class CoordinateType, int DIM>
class Wave_DataSource 
    : public DataSource<DataType,HandleType,CoordinateType,DIM>
{
  public:

    typedef int                                      OrdinalType;
    typedef Point<1,HandleType,CoordinateType>         PointType;
    typedef Teuchos::Comm<OrdinalType>               Communicator_t;
    typedef Teuchos::RCP<const Communicator_t>       RCP_Communicator;
    typedef Teuchos::RCP<Wave>                       RCP_Wave;

  private:

    // Wave object to operate on.
    RCP_Wave wave;

  public:

    Wave_DataSource(RCP_Wave _wave)
	: wave(_wave)
    { /* ... */ }

    ~Wave_DataSource()
    { /* ... */ }

    RCP_Communicator get_source_comm()
    {
	return wave->get_comm();
    }

    bool is_field_supported(const std::string &field_name)
    {
	bool return_val = false;

	if (field_name == "WAVE_SOURCE_FIELD")
	{
	    return_val = true;
	}

	return return_val;
    }

    bool is_local_point(const PointType &test_point)
    {
	bool return_val = false;

	if ( std::find(wave->get_grid().begin(), 
		       wave->get_grid().end(),
		       test_point.getCoords()[0] )
	     != wave->get_grid().end() )
	{
	    return_val = true;
	}

	return return_val;
    }

    const Teuchos::ArrayView<DataType> 
    get_source_data(const std::string &field_name)
    {
	Teuchos::ArrayView<DataType> return_view;

	if ( field_name == "WAVE_SOURCE_FIELD" )
	{
	    return_view = Teuchos::ArrayView<DataType>( wave->get_f() );
	}

	return return_view;
    }

    DataType get_global_source_data(const std::string &field_name)
    {
	DataType return_val = 0.0;

	return return_val;
    }
};
} // end namespace DataTransferKit
\end{lstlisting}

The {\sl DataTarget} interface implementation for the Wave code
follows the same principles as the {\sl DataSource} implementation
given by the previous listing. In this case, the Wave code is acting
as a target for the DAMPER\_FIELD and therfore all methods operating
by field name check for this field. In addition, the {\sl
  get\_target\_points} method recasts the one dimensional grid used by
the Wave code as a vector of three dimensional point objects, storing
them as member data. Again, the refactor methods are used here to
provide views into the private data structures of Wave.

\begin{lstlisting}
namespace DataTransferKit {

// DataTarget interface implementation for the Wave code.
template<class DataType, class HandleType, class CoordinateType, int DIM>
class Wave_DataTarget 
    : public DataTarget<DataType,HandleType,CoordinateType,DIM>
{
  public:

    typedef int                                      OrdinalType;
    typedef Point<1,HandleType,CoordinateType>       PointType;
    typedef Teuchos::Comm<OrdinalType>               Communicator_t;
    typedef Teuchos::RCP<const Communicator_t>       RCP_Communicator;
    typedef Teuchos::RCP<Wave>                       RCP_Wave;

  private:

    RCP_Wave wave;
    std::vector<PointType> local_points;

  public:

    Wave_DataTarget(RCP_Wave _wave)
	: wave(_wave)
    { /* ... */ }

    ~Wave_DataTarget()
    { /* ... */ }

    RCP_Communicator get_target_comm()
    {
	return wave->get_comm();
    }

    bool is_field_supported(const std::string &field_name)
    {
	bool return_val = false;

	if (field_name == "DAMPER_TARGET_FIELD")
	{
	    return_val = true;
	}

	return return_val;
    }

    const Teuchos::ArrayView<PointType> 
    get_target_points(const std::string &field_name)
    {
	Teuchos::ArrayView<PointType> return_view;

	if ( field_name == "DAMPER_TARGET_FIELD" )
	{
	    local_points.clear();
	    Teuchos::ArrayView<const DataType> local_grid( wave->get_grid() );
	    typename Teuchos::ArrayView<DataType>::const_iterator grid_it;
	    int n = 0;
	    int global_handle;
	    for (grid_it = local_grid.begin(); 
		 grid_it != local_grid.end();
		 ++grid_it, ++n)
	    {
		global_handle = wave->get_comm()->getRank() *
				local_grid.size() + n;
		local_points.push_back( 
		    point( global_handle, *grid_it) );
	    }
	    return_view = Teuchos::ArrayView<PointType>(local_points);
	}

	return return_view;
    }

    Teuchos::ArrayView<DataType> 
    get_target_data_space(const std::string &field_name)
    {
	Teuchos::ArrayView<DataType> return_view;

	if ( field_name == "DAMPER_TARGET_FIELD" )
	{
	    return_view = Teuchos::ArrayView<DataType>( wave->set_damping() );
	}

	return return_view;
    }

    void set_global_target_data(const std::string &field_name,
				const DataType &data)
    { /* ... */ }
};
} // end namespace DataTransferKit
\end{lstlisting}

%%---------------------------------------------------------------------------%%
\subsubsection{Damper Interface Implementations}
This section gives the two interface implementations for the Damper
code. The same principles used in the implementations of the Wave
interfaces also apply here. Again, the {\sl Damper\_Data\_Source}
implementation will only provide data on points that exist on its
grid. If they do not, no data will be applied. As the Damper is the
source for the DAMPER\_FIELD and a target for the WAVE\_FIELD, these
string values are used accordingly within the implementations.

\begin{lstlisting}
namespace DataTransferKit {

// DataSource interface implementation for the Damper code.
template<class DataType, class HandleType, class CoordinateType, int DIM>
class Damper_DataSource 
    : public DataSource<DataType,HandleType,CoordinateType,DIM>
{
  public:

    typedef int                                      OrdinalType;
    typedef Point<DIM,HandleType,CoordinateType>     PointType;
    typedef Teuchos::Comm<OrdinalType>               Communicator_t;
    typedef Teuchos::RCP<const Communicator_t>       RCP_Communicator;
    typedef Teuchos::RCP<Damper>                     RCP_Damper;

  private:

    // Damper object to operate on.
    RCP_Damper damper;

  public:

    Damper_DataSource(RCP_Damper _damper)
	: damper(_damper)
    { /* ... */ }

    ~Damper_DataSource()
    { /* ... */ }

    RCP_Communicator get_source_comm()
    {
	return damper->get_comm();
    }

    bool is_field_supported(const std::string &field_name)
    {
	bool return_val = false;

	if (field_name == "DAMPER_SOURCE_FIELD")
	{
	    return_val = true;
	}

	return return_val;
    }

    bool is_local_point(const PointType &test_point)
    {
	bool return_val = false;

	if ( std::find(damper->get_grid().begin(), 
		       damper->get_grid().end(),
		       test_point.getCoords()[0] )
	     != damper->get_grid().end() )
	{
	    return_val = true;
	}

	return return_val;
    }

    const Teuchos::ArrayView<DataType> 
    get_source_data(const std::string &field_name)
    {
	Teuchos::ArrayView<DataType> return_view;

	if ( field_name == "DAMPER_SOURCE_FIELD" )
	{
	    return_view = Teuchos::ArrayView<DataType>( damper->get_damping() );
	}

	return return_view;
    }

    DataType get_global_source_data(const std::string &field_name)
    {
	DataType return_val = 0.0;

	return return_val;
    }
};
} // end namespace DataTransferKit
\end{lstlisting}

As in the {\sl Wave\_Data\_Target} implementation, the Damper
implementation utilizes the {\sl Point} container to provide the
expected representation of its grid to the interfaces. Note that the
point container, and thus the interfaces, is templated on spatial
dimension.

\begin{lstlisting}
namespace DataTransferKit {

// DataTarget interface implementation for the Damper code.
template<class DataType, class HandleType, class CoordinateType, int DIM>
class Damper_DataTarget 
    : public DataTarget<DataType, HandleType, CoordinateType,DIM>
{
  public:

    typedef int                                      OrdinalType;
    typedef Point<1,HandleType,CoordinateType>       PointType;
    typedef Teuchos::Comm<OrdinalType>               Communicator_t;
    typedef Teuchos::RCP<const Communicator_t>       RCP_Communicator;
    typedef Teuchos::RCP<Damper>                     RCP_Damper;

  private:

    RCP_Damper damper;
    std::vector<PointType> local_points;

  public:

    Damper_DataTarget(RCP_Damper _damper)
	: damper(_damper)
    { /* ... */ }

    ~Damper_DataTarget()
    { /* ... */ }

    RCP_Communicator get_target_comm()
    {
	return damper->get_comm();
    }

    bool is_field_supported(const std::string &field_name)
    {
	bool return_val = false;

	if (field_name == "WAVE_TARGET_FIELD")
	{
	    return_val = true;
	}

	return return_val;
    }

    const Teuchos::ArrayView<PointType> 
    get_target_points(const std::string &field_name)
    {
	Teuchos::ArrayView<PointType> return_view;

	if ( field_name == "WAVE_TARGET_FIELD" )
	{
	    local_points.clear();
	    Teuchos::ArrayView<const DataType> local_grid( damper->get_grid() );
	    typename Teuchos::ArrayView<DataType>::const_iterator grid_it;
	    int n = 0;
	    int global_handle;
	    for (grid_it = local_grid.begin(); 
		 grid_it != local_grid.end();
		 ++grid_it, ++n)
	    {
		global_handle = damper->get_comm()->getRank() *
				local_grid.size() + n;
		local_points.push_back( 
		    point( global_handle, *grid_it) );
	    }
	    return_view = Teuchos::ArrayView<PointType>(local_points);
	}

	return return_view;
    }

    Teuchos::ArrayView<DataType> 
    get_target_data_space(const std::string &field_name)
    {
	Teuchos::ArrayView<DataType> return_view;

	if ( field_name == "WAVE_TARGET_FIELD" )
	{
	    return_view = Teuchos::ArrayView<DataType>( damper->set_wave_data() );
	}

	return return_view;
    }

    void set_global_target_data(const std::string &field_name,
				const DataType &data)
    { /* ... */ }
};
} // end namespace DataTransferKit
\end{lstlisting}

%%---------------------------------------------------------------------------%%
\subsubsection{Coupled Driver}
Once the interfaces have been implemented for both codes, the
infrastructure is in place to do a simple coupled problem. The
following listing gives the multiphysics driver used in the WaveDamper
problem.

After parallel initialization, a domain from 0.0 to 5.0 is distributed
in parallel across each process for each code. The bounds of the
computed local domain are applied in the constructor for each
code. Following this, the WAVE\_FIELD is setup for transfer. An
instance of the {\sl DataSource} interface is created from the {\sl
  Wave\_Data\_Source} implementation and {\sl DataTarget} interface
is created from the {\sl Damper\_Data\_Target} implementation. A {\sl
  DataField} is then constructed to transfer the WAVE\_FIELD from
the Wave code to the Damper code. The same process is repeated for the
DAMPER\_FIELD. 

Once the various Data Transfer Kit components have been initialized, an
iteration loop is setup to run until either the L2 norm of the wave
amplitude computed by the Wave code is less than 1.0E-6 or a specified
maximum number of iterations has been reached, each time executing the
algorithm specified by the picard iteration pseudocode above.

\begin{lstlisting}
#include <iostream>

#include "Wave.hpp"
#include "Wave_Source.hpp"
#include "Wave_Target.hpp"
#include "Damper.hpp"
#include "Damper_Source.hpp"
#include "Damper_Target.hpp"

#include <DataTransferKit_DataSource.hpp>
#include <DataTransferKit_DataTarget.hpp>
#include <DataTransferKit_CopyOperator.hpp>

#include "Teuchos_RCP.hpp"
#include "Teuchos_CommHelpers.hpp"
#include "Teuchos_DefaultComm.hpp"
#include "Teuchos_GlobalMPISession.hpp"

//---------------------------------------------------------------------------//
// Main function driver for the coupled Wave/Damper problem.
int main(int argc, char* argv[])
{
    // Setup communication.
    Teuchos::GlobalMPISession mpiSession(&argc,&argv);
    Teuchos::RCP<const Teuchos::Comm<int> > comm = 
	Teuchos::DefaultComm<int>::getComm();

    // Set up the parallel domain.
    double global_min = 0.0;
    double global_max = 5.0;
    int myRank = comm->getRank();
    int mySize = comm->getSize();
    double local_size = (global_max - global_min) / mySize;
    double myMin = myRank*local_size + global_min;
    double myMax = (myRank+1)*local_size + global_min;

    // Setup a Wave.
    Teuchos::RCP<Wave> wave =
	Teuchos::rcp( new Wave(comm, myMin, myMax, 10) );

    // Setup a Damper.
    Teuchos::RCP<Damper> damper =
	Teuchos::rcp( new Damper(comm, myMin, myMax, 10) ); 

    // Setup a Wave Data Source for the wave field.
    Teuchos::RCP<DataTransferKit::DataSource<double,int,double,1> > 
	wave_source = Teuchos::rcp( 
	    new DataTransferKit::Wave_DataSource<double,int,double,1>(wave) );

    // Setup a Damper Data Target for the wave field.
    Teuchos::RCP<DataTransferKit::DataTarget<double,int,double,1> > 
	damper_target = Teuchos::rcp( 
	    new DataTransferKit::Damper_DataTarget<double,int,double,1>(damper) );

    // Setup a copy operator for the wave field.
    DataTransferKit::CopyOperator<double,int,double,1> 
	wave_field_op( comm,
		       "WAVE_SOURCE_FIELD",
		       "WAVE_TARGET_FIELD",
		       wave_source,
		       damper_target );

    // Setup a Damper Data Source for the damper field.
    Teuchos::RCP<DataTransferKit::DataSource<double,int,double,1> > 
	damper_source = Teuchos::rcp( 
	    new DataTransferKit::Damper_DataSource<double,int,double,1>(damper) );

    // Setup a Wave Data Target for the damper field.
    Teuchos::RCP<DataTransferKit::DataTarget<double, int, double,1> > 
	wave_target = Teuchos::rcp( 
	    new DataTransferKit::Wave_DataTarget<double,int,double,1>(wave) );

    // Setup a copy operator for the damper field.
    DataTransferKit::CopyOperator<double,int,double,1> 
	damper_field_op( comm,
			 "DAMPER_SOURCE_FIELD",
			 "DAMPER_TARGET_FIELD",
			 damper_source,
			 wave_target );

    // Iterate between the damper and wave until convergence.
    double local_norm = 0.0;
    double global_norm = 1.0;
    int num_iter = 0;
    int max_iter = 100;

    // Create the mapping for the wave field.
    wave_field_op.create_copy_mapping();

    // Create the mapping for the damper field.
    damper_field_op.create_copy_mapping();

    while( global_norm > 1.0e-6 && num_iter < max_iter )
    {
	// Transfer the wave field.
	wave_field_op.copy();

	// Damper solve.
	damper->solve();

	// Transfer the damper field.
	damper_field_op.copy();

	// Wave solve.
	local_norm = wave->solve();

	// Collect the l2 norm values from the wave solve to ensure
	// convergence. 
	Teuchos::reduceAll<int>( *comm,
				 Teuchos::REDUCE_MAX, 
				 int(1), 
				 &local_norm, 
				 &global_norm );

	// Update the iteration count.
	++num_iter;

	// Barrier before proceeding.
	Teuchos::barrier<int>( *comm );
    }

    // Output results.
    if ( myRank == 0 )
    {
	std::cout << "Iterations to converge: " << num_iter << std::endl;
	std::cout << "L2 norm:                " << global_norm << std::endl;
    }

    return 0;
}
\end{lstlisting}

The problem was noted to converge for all cases tested in 22 iterations.

%%---------------------------------------------------------------------------%%
\subsection{WaveDamper Testing}

Using the WaveDamper example, the Data Transfer Kit parallel
capabilities have been demonstrated to operate with a variety of
hardware and software components. Hardware tested includes shared
memory local workstations, the CASL fissile four machines, and a small
distributed memory cluster on the University of Wisconsin - Madison
campus. Both Intel and GNU compilers have been tested along with both
MPICH2 and OpenMPI implementations of the MPI standard. Successful
execution of the WaveDamper test was achieved on 120 cores on the
small distributed memory cluster.

%%---------------------------------------------------------------------------%%
\section{Conclusion}
An initial design and implementation of the Data Transfer Kit has been
completed. This package attempts to define an interface for data
transfer to meet the stated design requirements along with an
implementation for parallel operations. This simple interface and
implementation should serve as a more general starting point for
advanced mesh-based coupling and data structures.

The Data Transfer Kit is currently capable of a simple form of point
based mesh coupling. The software infrastructure used in the
development of the Data Transfer Kit, instructions on obtaining,
building, and testing the Data Transfer Kit, and an outline of how to
use the package presented through a concrete example have been
provided as an outline for this work. The WaveDamper example has been
demonstrated to operate in parallel on various hardware elements from
the desktop level to a small scale development cluster, including the
CASL fissile four development machines. 

%%---------------------------------------------------------------------------%%
\pagebreak
\bibliographystyle{ieeetr}
\bibliography{references}
\end{document}


